\documentclass[12pt,letterpaper]{article}
\usepackage{hyperref}

\begin{document}
\title{Pilot Study in the Evaluation of Mood and Motor Functions in Parkinson's Disease Through a Computer Keyboard Online Clinical Telegaming Simulation}
\author{Linda Xu, Stephan Loh, CTGaming Team Members,Yuri Bonilla}
\maketitle
 
\section{Abstract}
 Introduction: Parkinson's Disease is a neurodegenerative disorders that affects the lives of people in middle age. Parkinson's Disease interferes with their daily living through a decline in movement and experiencing emotional issues that relate to not being able to live a stable lifestyle. The study will focus on improved mood as a result of using an online gaming system over the seven day period. 
 Materials and Methods: Patient volunteers that have Parkinson's Disease and patient volunteers that do not have Parkinson's Disease from the San Diego area between the ages of 60 and 80 years of age will be recruited for the study and participate in the study for a period of seven days. 
 Results: will be added as results are released to the team
 Discussion: will be added as results are released to the team
 
\section{Introduction}
 Parkinson's Disease is a neurodegenerative disorder that affects mood and motor functions. The quality of life of Parkinson's Disease patients decreases as the disorder progresses to the advanced stages in the long term. Parkinson's Disease consists of physical symptoms, such as slowness in motor activity, muscular stiffness, not being able to move, stooped posture, abnormal facial expressions, gait shuffling, tremor, struggle with walking, and not maintaining bodily balance\cite{Jenkinson2011}. Over time, advanced stages of Parkinson's Disease affects the quality of life in completing daily tasks, mental activity, emotional activity, and social activity in daily interactions with oneself and others \cite{Jenkinson2011}. Parkinson's Disease typically affects individuals in the age group of 65 years of age and older \cite{Serrano-Duenas2008}. Parkinson's Disease is associated with mood related health conditions, which include anxiety, depression, and fatigue \cite{Wang2014}. The study will focus on being active in the online game that will require attention to focus on the game by applying keyboard functions. The online game will allow for the study to focus on finding out the mood and motor function reaction times in the patient volunteers that have Parkinson's Disease and patient volunteers who do not have Parkinson's Disease while implementing the questionnaire that will be delivered as part of the selection criteria and addressing mood and motor function. The study will have a pilot study sample size of Parkinson's Disease and non Parkinson's Disease patient volunteers that will include both male and female. The questionnaire before and after the clinical telegaming project will be provided to the patient volunteers after the clinical telegaming study to evaluate the mood and motor function components that the study is addressing. This study will determine whether it is helpful, harmful, or does not have an effect on addressing mood and motor functions in relation to Parkinson's Disease, the neurodegenerative disorder, in participating in the computer keyboard clinical telegaming simulation pilot trial. 
 
\section{Materials and Methods}
 In order for the study to be completed successfully, it is necessary to have a plan for the methods of the study to be completed. The necessary items for the study will include a computer with a keyboard that allows for access to high speed internet in the patient volunteer's home, patient information questions and forms delivered electronically for the inclusion and exclusion factors, including the protection of privacy, patient volunteer information in a computer to track activity and response on mood and motor functions before and after the study, and time to commit to the pilot trials to be conducted by the six team members of the BHA, also known as the Brain Health Alliance, throughout the United States of America once daily within the five day period. Informed consent will be delivered through the BHA website as a way to inform prospective patients about what they will be involved in the study before they proceed even further. 
 
 There will be no financial cost to go through the study, since the questionnaire is simple and user friendly, which will not require any out of pocket cost on the BHA team's behalf. The computer with a keyboard and access to high speed internet will be the patient volunteer's responsibility as part of the requirements to participate and complete the trial. However, the primary requirement will be a small amount the patient volunteer's  time throughout the five day trial(How many minutes per day will the patient volunteers spend in the pilot trial?).
 
 The exclusion criteria for the  includes people who have disabilities, such as blindness, unable to follow instructions without understanding what is to be done during the pilot trial after given thorough instructions during each step of the project, long term inability to move their body, such as hemiplegia, cognitive impairment that conflicts with communicating answers to the questionnaires in the cases of advanced stages of a neurodegenrative disorder such as Parkinson's Disease, and no giver involvement for someone that can still communicate some of the information about their health \cite{Martinez-Martin2007}. In addition, prospective patient volunteers will be excluded if the following conditions that can affect the consistency of the project are present and affect the ability to be in the pilot study, such as deafness and missing limbs. Other exclusion factors include severe hypertension, heart issues, seizures, involvement in another study that relates to physical activity, involvement in pharmaceutical drug research studies, orthopedic conditions, and Alzheimer's Disease, a neurodegenerative disorder \cite{Pompeua2014}.
 
 Selection criteria will include patient volunteers that are able to commit once daily during the five day period in the pilot study in an online format by using the computer with a keyboard, be between the ages of 60 and 80, be either a Parkinson's Disease patients or non Parkinson's Disease patients, and can follow instructions when requested throughout the study on tasks, such as answering the questionnaires before and after the study and committing five days to following through the keyboard buttons to complete the task that is being requested, in order to maintain consistency in the results of the study. 
  
 The delivery of the questionnaires for recruited prospective patient volunteers will occur as soon as the Brain Health Alliance puts up the questionnaires for the selection process on the Brain Health Alliance website delivered via online. The initial questionnaire will be general health status questions that relate to mood and motor functions of daily life that will be of benefit to the selection criteria. The HRQOL, also known as the Health Related Quality of Life, and the SF 12, also known as the ,will be used to analyze physical and mental health. Both questionnaire inventories will be used for the study, since it will measure the state of health for both Parkinson's Disease patients and Non-Parkinson's Disease patients. 
 
 The HRQOL and SF-12 questionnaires will be validated through two types of validation , which include content validity and construct validity. Through content validity, the questions in both the HRQOL and SF- 12 were chosen for the purposes of addressing motor and mood functions in the Parkinson's Disease and Non- Parkinson's Disease patient volunteers along with the selection criteria through a mixture of subjective and objective questions. At the same time, content validity in the HRQOL and the SF- 12 questionnaires will have selected questions that are relevent to the study of mood and motor function of patients that have Parkinson's Disease and patients that do not have any neurodegenerative disorders\cite{jenkinson2011}. Both the HRQOL and the SF-12  will use construct validity to measure the trends og motor and mood functions of patients with Parkinson's Disease and patients without a neurodegenerative disorder with respect to complying with the hypothesis of mood and motor functions in participating in the clinical telegaing study of using a computer and keyboard \cite{jenkinson2011}. 

 After the selection process and agreement from the selected volunteer patients is completed, it will be necessary to initiate the trial for once a day for five days during a scheduled time that medication is not being taken for neurodegenerative disorders. 
 
 The clinical telegaming team will be reaching out to Parkinson's Disease communities that are part of an organization in the San Diego area in the pilot study that is consistent with travel time, time to complete
 the trials in the study,location consistency, and money. The questionnaire will include selection criteria information, such as age, gender, schedule for being able to participate in the study, location, and other important components that are yet to be determined for validity of the study as well as health state questions \cite{Jenkinson2011}. The team will be searching for  via phone, in person, and/or via email. We will also recruit healthy patients in the San Diego area via phone, in person, and/or via email. 
 
 The initial address to the questionnaire will include the purpose for the questionnaire, and it will be addressed as a non- harmful study with respect to observing the improvement of mood and motor function, more severe reactions to mood and motor function, or no changes in mood and motor functions, of patients with and without Parkinson's Disease, as what are included in the hypothesis. The online game
 will be done in the comfort of patient volunteer homes during the actual study to test the movements and mood of patient volunteers. 
 
 The plan is to have once daily keyboard game simulation when it does not take place with neurologically involved pharmaceuticals being taken by the Parkinson's Disease patient volunteers involved in the study. The study will be taking place once a day for five days in the weekday. 
 
 The questionnaire breakdown:
 
 
 Are there any risks for the keyboard game?
 
 Will the patient volunteers be trained to complete the study?
 
 Description of the keyboard based game with actions?
 
\section{Results}
 Patient volunteers and healthy patient volunteer controls actual demographics
 Answer to the medical/scientific question?
 SF 12 and HRQOL Questionnaire Before and After the keyboard gaming simulation Data Analysis 
 Reaction Times during the five day trial
 Include results in tables, graphs, charts tp make the study more revealing to review
 Include standard deviation, mean difference
 
\section{Discussion}
 Evaluation Discussion based on the results...
 How was each keyboard gaming simulation day during the five day period?
 Was there any pharmaceutical drug interaction while the patient volunteers that participated in the study?
 Was the keyboard study effective?
 What was done effectively in the study?
 How did HRQOL and SF 12 Questionnaires relate toreaction time with respect to mood and motor functions?
 Are the questionnaires effective in addressing mood and motor health statuses of patients?
 Discuss previous studies and connect with this study...
 Are there any improvements to be made? The study can improve by having a larger sample size to make the study more of a clinical trial. At the same time, having a larger sample size would have allowed for results that would have consistency with comparing results. Recruiting for a larger sample size with a Parkinson's Disease organization from one location in prior months before the study with specific written communication about the proposed specific study and expectations in the informed consent needed for the prospective patient volunteers with Parkinson's Disease and the controls to be notified bout before deciding to participate in the study. It would give more time for the administrative staff of the Parkinson's Disease association in a select area to communicate with association affiliated prospective patient volunteers to participate in the study. 

\nocite{*}
\bibliographystyle{plain}
\bibliography{YBonilla_BHA_References_2014}

\end{document}
