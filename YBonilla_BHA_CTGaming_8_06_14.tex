\documentclass[12pt,letterpaper]{article}
\usepackage{hyperref}

\begin{document}
\title{Pilot Study in the Evaluation of Mood and Motor Functions in Parkinson's Disease through an Online Clinical Telegaming Simulation}
\author{Linda Xu, Stephan Loh, CTGaming Team Members,Yuri Bonilla}
\maketitle
 
\section{Abstract}
 Introduction: Parkinson's Disease is a neurodegenerative disorders that affects the lives of people in middle age. Parkinson's Disease interferes with their daily living through a decline in movement and experiencing emotional issues that relate to not being able to live a stable lifestyle. The study will focus on improved mood as a result of using an online gaming system over the seven day period. 
 Materials and Methods: Patient volunteers that have Parkinson's Disease and patient volunteers that do not have Parkinson's Disease from the San Diego area between the ages of 60 and 80 years of age will be recruited for the study and participate in the study for a period of seven days. 
 Results:
 Discussion:
 
\section{Introduction}
 Parkinson's Disease is a neurodegenerative disorder that affects mood and motor functions. The quality of life of Parkinson's Disease patients decreases as the disorder progresses to the advanced stages in the long term. Parkinson's Disease consists of physical symptoms, such as slowness in motor activity, muscular stiffness, not being able to move, stooped posture, abnormal facial expressions, gait shuffling, tremor, struggle with walking, and not maintaining bodily balance\cite{Jenkinson2011}. Over time, advanced stages of Parkinson's Disease affects the quality of life in completing daily tasks, mental activity, emotional activity, and social activity in daily interactions with oneself and others \cite{Jenkinson2011}. Parkinson's Disease typically affects individuals in the age group of 65 years of age and older \cite{Serrano-Duenas2008}. Parkinson's Disease is associated with mood related health conditions, which include anxiety, depression, and fatigue \cite{Wang2014}. The study will focus on being active in the online game that will require attention to focus on the game. The online game will allow for the study to focus on finding out the mood and motor function reaction times in the patient volunteers that have Parkinson's Disease and patient volunteers who do not have Parkinson's Disease while implementing the questionnaire that will be delivered as part of the selection criteria and addressing mood and motor function. The study will have a pilot study sample size of Parkinson's Disease and non Parkinson's Disease patient volunteers that will include both male and female. This will be part of the selection process for the pilot clinical telegaming rehabilitation and intervention system study through an online gaming system. The questionnaire after the clinical telegaming project will be provided to the patient volunteers after the clinical telegaming study to evaluate the mood and motor function components that the study is attempting to address. This study will determine whether it is helpful, harmful, or does no effect on addressing mood and motor functions in relation to the neurodegenerative disorder, which Parkinson's Disease will be the focus throughout the execution of the study.
 
\section{Materials and Methods}
 In order for the study to be completed successfully, it is necessary to have a plan for the methods of the study to be completed. The necessary items for the study will include a computer that allows for access to high speed internet in the patient volunteer's home, patient information questions and forms delivered electronically for the inclusion and exclusion factors, including the protection of privacy, patient volunteer information in a computer to track activity and response on mood and motor functions before and after the study, and time to commit to the pilot trials to be conducted by three team members of the BHA, also known as the Brain Health Alliance, in the San Diego, California area in the United States of America once daily within the seven day period. Informed consent will be delivered through the BHA website as a way to inform prospective patients about what they will be involved in the study before they proceed even further. 
 
 There will be no financial cost as a questionnaire is simple and doesn't require any money to make. The only requirement is a small amount the patient's time.
 
 The exclusion criteria for the  includes people who have disabilities, such as blindness, unable to follow instructions without understanding what is to be done during the pilot trial after given thorough instructions during each step of the project, long term inability to move their body, such as hemiplegia, cognitive impairment that conflicts with communicating answers to the questionnaires in the cases of advanced stages of a neurodegenrative disorder such as Parkinson's Disease, and no caregiver involvement for someone that can still communicate some of the information about their health \cite{Martinez-Martin2007}. In addition, prospective patient volunteers will be excluded if they present with the following conditions that can affect the consistency of the project, which include deafness and missing limbs, which missing limbs will not be useful enough to target motor function in terms of movement. Other exclusion factors are severe hypertension that interferes with bodily movement control, unstable angina, recent myocardial infraction, seizures, involvement in another study that relates to physical activity,  pharmaceutical drug research, orthopedic conditions, and altering neurological disorders \cite{Pompeua2014}.
 
 Selection criteria will include patient volunteers that are able to commit once daily during the seven day period in the pilot study in an online format by using the computer, have the ages between 60 and 90 years of age, will include Parkinson's Disease patients and non Parkinson's Disease patients, and can follow instructions when requested to do so in order to maintain consistency in the results of the study. 
  
 The delivery of the questionnaires for prospective patient volunteers will occur as soon as the Brain Health Alliance puts up the questionnaires for the selection process on the Brain Health Alliance website delivered via online. The initial questionnaire will be general health status questions that will help with the selection criteria. The HRQOL will be used for the study, since it measures state of health for both Parkinson's Disease patients and Non-Parkinson's Disease patients. It will also be used with SF 12 to specifically analyze physical and mental health more critically. 

 After the selection process and agreement from the selected volunteer patients is completed, it will be necessary to initiate the trial for at least two weeks. It will be done twice a week at minimum.
 
 The Clinical Telegaming team will be reaching out to Parkinson's Disease communities that are part of an organization in the San Diego area in the pilot study that is consistent with travel time, time to complete
 the trials in the study,location consistency, and money. The questionnaire will include selection criteria information, such as age, gender, schedule for being able to participate in the study, location, and other important components that are yet to be determined for validity of the study as well as health state questions \cite{Jenkinson2011}. The team will address to the PD communities about the pilot study as a way for them to recruit actual patients volunteers that have PD by communicating with the coordinators of the PD community organizations in the San Diego area either via phone, in person, and/or via email. We will also recruit healthy patients in the San Diego area via phone, in person, and/or via email. 
 
 The initial address to the questionnaire will include the purpose for the questionnaire, and it will be addresses as a non- harmful test with respect to observing the improvement of mood and motor function, more severe reactions to mood and motor function, or no changes in mood and motor functions, of patients with and without Parkinson's Disease, as what are included in the hypothesis. The online game
 will be taken to the patient volunteer homes during the actual study to test the movements and mood of patient volunteers. We are thinking of completing the study at least twice per week for a three week period due time that we have left to complete the study in the San Diego area. 
 
 The plan is to have once daily keyboard game simulation when it does not take place with neurologically involved pharmaceuticals being taken by the Parkinson's Disease patient volunteers involved in the study. The study will be taking place for five days in the weekday. 
 
\section{Results}
 Outcomes of Evaluation	
 Include results in tables, graphs, charts
 
\section{Discussion}
 Evaluation Discussion
 Was the keyboard study effective?
 Are the questionnaires effective in addressing mental health status of patients?
 The study can improve by having a larger sample size to make the study more of a clinical trial. At the same time, having a larger sample size would have allowed for results that would have consistency with comparing results. Recruiting for a larger sample size with a Parkinson's Disease organization from one location in prior months before the study with specific written communication about the proposed specific study and expectations in the informed consent needed for the prospective patient volunteers with Parkinson's Disease and the controls to be notified bout before deciding to participate in the study. It would give more time for the administrative staff of the Parkinson's Disease association in a select area to communicate with association affiliated prospective patient volunteers to participate in the study. 

\nocite{*}
\bibliographystyle{plain}
\bibliography{YBonilla_BHA_References_2014}

\end{document}
